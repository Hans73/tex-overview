\documentclass{scrartcl}

\usepackage{
graphicx,
cooltooltips,
dtklogos,
tikz,
libertine
%xltxtra
}

\def\XeT{XET}
\def\Aleph{Aleph}
\def\XeLaTeX{Xe\LaTeX}
\def\ConTeXt{Con\TeX{}t}
%\setmainfont{Linux Libertine}

\def\mynode#1#2#3#4{
  \node (#1) at (#2) {
    \cooltooltip{#1}{#3}{#3}{}{#4\strut}
  };
}
\title{A short overview of \TeX\ and its children\dots}
\author{Arno Trautmann\thanks{arno.trautmann@gmx.de -- Please feel free to mail me any suggestions and comments!}}
\date{}
\begin{document}
\maketitle

This paper tries to give a short overview of the development of \TeX. So far, most of the information is from the article \textsf{A brief history of \TeX, volume II} by Arthur Reutenauer in \textsf{EuroBacho\TeX 2007}.

\section*{\TeX\ -- the program}
\begin{tikzpicture}
\mynode{tex}{7,2}{born in 1978}{\TeX}

\mynode{xet-tex}{3,1}{The first extension to TeX, 1987. It was able to typeset in two directions, but only with a mark in the dvi to change the direction.}{\TeX-\XeT}
\draw (tex) to (xet-tex);

\mynode{xet--tex}{3,0}{TeX--XeT was able to really put the glyphs on the right place in the dvi.}{\TeX-\relax{}-\XeT}
\draw (xet-tex) to (xet--tex);

\mynode{tex3}{7,0}{Now handling 8-bit input. 1989}{\TeX3};
\draw (tex) to (tex3);

\mynode{omega}{2,-2}{Support for unicode-input. Still constrained on the output}{$\Omega$};
\draw (tex3) to (omega);

\mynode{etex}{4,-2}{*the* extension to TeX.}{$\varepsilon$-\TeX};
\draw (xet--tex) to (etex);

\mynode{pdftex}{6,-2}{A new engine to directly produce pdf-files from TeX, without the need of dvi-ps-pdf. This allows to use microtypographic extensions and many other features of the pdf format.}{pdf\TeX};
\draw (tex3) to (pdftex);

\mynode{texgx}{8,-2}{?}{\TeX{}GX};
\draw (tex3) to (texgx);

\mynode{nts}{10,-2}{A project to completely reimplement TeX in Java. Now NTS is officially declared dead.}{\NTS};
\draw (tex3) to (nts);

\mynode{aleph}{3,-4}{originally named epsilon-Omega, an attempt to stabilize Omega while merging epsilon extensions.}{\Aleph};
\draw (omega) to (aleph);
\draw (etex) to (aleph);

\mynode{xetex}{8,-4}{Originally started as a project for Mac OS, this quite young extension enables extremely multilingual support for left-to-right typesetting, right-to-left and almost any other possible directio. XeTeX also features support for OpenType and AAT-fonts.}{\XeTeX};
\draw (texgx) to (xetex);
\draw (etex) to (xetex);

\mynode{extex}{10,-4}{?!}{$\epsilon\chi$\TeX};
\draw (nts) to (extex);

\mynode{pdfetex}{5,-4}{Merging the pdfTeX engine with the eTeX-extensions. This engine can produce dvi (with or without the eTeX-extensions) as well as pdf (again, with or without extensions).}{pdf($\epsilon$)-\TeX};
\draw (etex) to (pdfetex);
\draw (pdftex) to (pdfetex);

\mynode{luatex}{4,-6}{Still in heavy active development, LuaTeX will support unicode, OpenType and totally everything. It features an embedded scripting language, lua, making it easy to extend.}{Lua\TeX};
\draw (aleph) to (luatex);
\draw (pdfetex) to (luatex);

\mynode{vtex}{0,-8}{Please insert information here...}{V\TeX};
\mynode{revtex}{2,-8}{Please insert information here...}{REV\TeX};

\end{tikzpicture}
\newpage

\section*{\LaTeX\ -- a large macro package for \TeX}
\begin{tikzpicture}
\mynode{latex209}{0,0}{The first official version by Leslie Lamport, 1985. It's called "Lamport's TeX".}{\LaTeX\ 2.09};

\mynode{amslatex}{2,-2}{A variation of LaTeX2.09}{\AMS\LaTeX};
\draw (latex209) to (amslatex);

\mynode{slitex}{4,-2}{A variation of LaTeX2.09}{SLI\TeX};
\draw (latex209) to (slitex);

\mynode{latex2ε}{0,-4}{New release of LaTeX to avoid incompatible dialects of LaTeX 2.09.}{\LaTeXe};
\draw (latex209) to (latex2ε);
\draw (amslatex) to (latex2ε);
\draw (slitex) to (latex2ε);

\mynode{pdflatex}{2,-6}{}{pdf\LaTeXe};
\draw (latex2ε) to (pdflatex);

\mynode{xelatex}{4.5,-6}{Using the XeTeX engine. There are some special packages that provide easy access to the modern features of XeTeX.}{\XeLaTeX};
\draw (latex2ε) to (xelatex);

\mynode{lualatex}{7,-6}{LaTeX based on LuaTeX with dvi or pdf output.}{(pdf)Lua\LaTeX};
\draw (latex2ε) to (lualatex);

\mynode{lambda}{-2,-6}{A LaTeX-package for omega.}{$\Lambda$}
\draw (latex209) to (lambda);

\mynode{latex3}{0,-8}{}{\LaTeX{}3};
\draw (latex2ε) to (latex3);
\end{tikzpicture}
\newpage

\section*{\ConTeXt\ -- the other major \TeX\ macro package}
\begin{tikzpicture}
\mynode{mki}{0,0}{Original ConTeXt with Dutch low level interface.}{\ConTeXt MkI};
\mynode{mkii}{0,-2}{ConTeXt with English low level interface. Works with any TeX-engine, like LaTeX: TeX, e-TeX, pdfTeX, Aleph, XeTeX, ...}{\ConTeXt MkII};
\draw (mki) to (mkii);

\mynode{mkiii}{0,-4}{Reserved for future use for files supporting XeTeX.}{\ConTeXt\ MkIII};
\draw (mkii) to (mkiii);

\mynode{mkiv}{0,-6}{Specially designed for LuaTeX.}{\ConTeXt MkIV};
\draw (mkiii) to (mkiv);
\end{tikzpicture}


\end{document}